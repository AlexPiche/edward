% Define the subtitle of the page
\title{Point-based evaluations}

% Begin the content of the page
\subsection{Point-based evaluations}

A point-based evaluation is a scalar-valued metric for assessing
trained models. For example, we can assess models for classification
by predicting the label for each observation in the data and comparing
it to their true labels. Edward implements a variety of metrics, such
as classification error and mean absolute error.

Formally, point prediction in probabilistic models is given by
taking the mean of the posterior predictive distribution,
\begin{align*}
  p(x_\text{new} \mid x)
  &=
  \int
  p(x_\text{new} \mid z)
  p(z \mid x)
  \text{d} z.
\end{align*}
The model's posterior predictive can be used to generate new data
given past observations and can also make predictions on new data
given past observations.
It is formed by calculating the likelihood of the new data, averaged
over every set of latent variables according to the posterior
distribution.

\subsubsection{Implementation}

Edward implements point-based evaluations through the
\texttt{predict()} method in the probability model. It predicts the
label given samples from the posterior $p(z \mid x)$, i.e., it is the
mean of $p(x_\text{new} \mid z)$ averaged over posterior samples.
\begin{lstlisting}[language=Python]
class BayesianLinearRegression:
    ...
    def predict(self, xs, zs):
        """Return a prediction for each data point, averaging over
        each set of latent variables z in zs."""
        x_test = xs['x']
        b = zs[:, 0]
        W = tf.transpose(zs[:, 1:])
        y_pred = tf.reduce_mean(tf.matmul(x_test, W) + b, 1)
        return y_pred

model = BayesianLinearRegression()
\end{lstlisting}
Similar to variational inference methods, the
\texttt{ed.evaluate()} method takes as input the probability model,
variational model, and a data set. It evaluates the trained model (the
posterior via the variational model; everything else via the
probability model) according to a chosen metric.
\begin{lstlisting}[language=Python]
ed.evaluate('categorical_accuracy', model, variational, data)
ed.evaluate('mean_absolute_error', model, variational, data)
\end{lstlisting}
The \texttt{data} can be data held-out from training time, making it
easy to implement cross-validated techniques.

Point-based evaluation applies generally to any setting, including
unsupervised tasks. For example, we can evaluate the likelihood of
observing the data.
\begin{lstlisting}[language=Python]
ed.evaluate('log_likelihood', model, variational, data)
\end{lstlisting}

Point-based evaluations are formally known as scoring rules
in decision theory. Scoring rules are useful for model comparison, model
selection, and model averaging.

See the \href{api/criticisms.html}{criticism API} for further details.
An example of point-based evaluation is in the
\href{tut_supervised_regression.html}{supervised learning
(regression)} tutorial.

\subsubsection{References}\label{references}

\begin{itemize}
\item
  Gneiting, T., & Raftery, A. E. (2007). Strictly Proper Scoring
  Rules, Prediction, and Estimation. Journal of the American
  Statistical Association, 102(477), 359–378.
\end{itemize}
